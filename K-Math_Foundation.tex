\documentclass[11pt]{article}
\usepackage{amsmath,amsthm,amssymb,amsfonts,mathtools}
\usepackage[hidelinks]{hyperref}
\usepackage[margin=1in]{geometry}
\title{Foundations of K-Math: Algebraic Structure, Spectral Theory, and Conditional Cryptanalytic Implications}
\author{Brendon Joseph Kelly \\ K Systems and Securities}
\date{August 23, 2025}
\theoremstyle{plain}
\newtheorem{theorem}{Theorem}[section]
\newtheorem{lemma}[theorem]{Lemma}
\newtheorem{corollary}[theorem]{Corollary}
\theoremstyle{definition}
\newtheorem{definition}[theorem]{Definition}
\newtheorem{axiom}[theorem]{Axiom}
\newtheorem{example}[theorem]{Example}
\theoremstyle{remark}
\newtheorem{remark}[theorem]{Remark}
\DeclareMathOperator{\supp}{supp}
\DeclareMathOperator{\diag}{diag}
\begin{document}
\maketitle
\begin{abstract}
We formalize the core algebraic and analytic objects of \emph{K-Math}: a fusion operator $\ast$ acting on a finite-dimensional complex vector space endowed with a distinguished basis indexed by an underlying finite abelian group $G$. We prove that, under natural axioms, $(\mathbb{C}^G, \ast)$ forms a unital, associative, commutative Banach \emph{convolution-type} algebra admitting a unitary diagonalizing transform (the \emph{K-transform}). We establish norm submultiplicativity, spectral factorization for positive-definite elements, and a Plancherel identity. Finally, we state \emph{conditional} complexity-theoretic implications: if the K-transform satisfies a dimension/sparsity collapse property (K-DRP), then one obtains subexponential improvements for certain hard problems (e.g., decisional LWE at specific parameter regimes, or discrete log in some groups) \emph{relative to those assumptions}. These results motivate \emph{constructive} post-quantum designs while avoiding prescriptive attack procedures.
\end{abstract}
\section{Introduction}
Modern cryptography rests on structured algebra (groups, rings, lattices) and transforms (Fourier, Walsh--Hadamard, number-theoretic transforms). K-Math introduces a fusion operator $\ast$, intended to combine signals, keys, or states in a way that is algebraically tractable yet spectrally expressive. Our goals are threefold:
\begin{enumerate}
    \item Specify axioms that make $(\mathbb{C}^G, \ast)$ a well-behaved algebra.
    \item Construct a unitary transform $\mathcal{K}$ that diagonalizes $\ast$ (turning $\ast$ into pointwise multiplication).
    \item Explore \emph{conditional} implications of additional hypotheses (K-DRP) for complexity.
\end{enumerate}
Throughout, $G$ denotes a finite abelian group written additively; $\mathbb{C}^G$ denotes complex functions on $G$.
\section{Core Definitions and Axioms}
\begin{definition}[K-Math ambient space]
Let $V := \mathbb{C}^G$ with inner product $\langle f, h \rangle := \sum_{x \in G} f(x) \overline{h(x)}$ and induced $\ell_2$ norm $\lVert f \rVert_2$.
\end{definition}
\begin{definition}[Fusion operator]\label{def:fusion}
Fix a \emph{kernel} $k \in \mathbb{C}^G$ with $k(0) \neq 0$. Define for $f,h \in V$:
\[
    (f \ast h)(x) := \sum_{y \in G} f(y) h(x - y) k(y).
\]
We call $\ast$ the \emph{K-Math fusion operator} with kernel $k$.
\end{definition}
\begin{remark}
When $k \equiv 1$, $\ast$ reduces to the standard group convolution. Allowing a general $k$ yields a weighted (or ``twisted'') convolution. This model captures the stated K-Math intuition while remaining rigorously analyzable.
\end{remark}
We impose the following axioms, all satisfied by Definition~\ref{def:fusion} on finite abelian $G$.
\begin{axiom}[Bilinearity] $\ast$ is bilinear in both arguments.
\end{axiom}
\begin{axiom}[Commutativity] $f \ast h = h \ast f$ for all $f,h \in V$.
\end{axiom}
\begin{axiom}[Associativity] $(f \ast h) \ast g = f \ast (h \ast g)$ for all $f,h,g \in V$.
\end{axiom}
\begin{axiom}[Unitality] There exists $e \in V$ such that $f \ast e = e \ast f = f$ for all $f \in V$.
\end{axiom}
\begin{axiom}[Norm control] There exists $C_k > 0$ (depending only on $k$) with $\lVert f \ast h \rVert_2 \le C_k \lVert f \rVert_2 \lVert h \rVert_1$.
\end{axiom}
\section{Existence and Form of the Identity}
\begin{lemma}[Identity element]\label{lem:identity}
Let $\delta_0 \in V$ be the Dirac at $0 \in G$. Set $e := \delta_0 / k(0)$. Then for all $f \in V$, $(f \ast e)(x) = f(x)$ and $(e \ast f)(x) = f(x)$.
\end{lemma}
\begin{proof}
Compute
\[
    (f \ast e)(x) = \sum_{y \in G} f(y) e(x - y) k(y) = \sum_{y \in G} f(y) \delta_0(x - y) \frac{k(0)}{k(0)} = f(x).
\]
The other identity is analogous by commutativity.
\end{proof}
\section{Associativity and Commutativity}
\begin{lemma}[Commutativity]\label{lem:comm}
For all $f,h \in V$,
\[
    (f \ast h)(x) = \sum_{y} f(y) h(x - y) k(y) = \sum_{z} h(z) f(x - z) k(z) = (h \ast f)(x).
\]
\end{lemma}
\begin{theorem}[Associativity]\label{thm:assoc}
For all $f,h,g \in V$, $(f \ast h) \ast g = f \ast (h \ast g)$.
\end{theorem}
\begin{proof}
Expanding both sides yields
\[
    ((f \ast h) \ast g)(x) = \sum_{u \in G} (f \ast h)(u) g(x-u) k(u) = \sum_{u \in G} \sum_{y \in G} f(y) h(u-y) k(y) g(x-u) k(u).
\]
Changing variables $z := u - y$ gives
\[
    ((f \ast h) \ast g)(x) = \sum_{y,z \in G} f(y) h(z) g(x-y-z) k(y) k(y+z).
\]
Similarly,
\[
    (f \ast (h \ast g))(x) = \sum_{y,z \in G} f(y) h(z) g(x-y-z) k(z) k(y).
\]
Thus associativity holds whenever $k(y+z) = k(y)$ for all $y,z$, which is the case when $k$ is constant. To incorporate nontrivial kernels, we absorb $k$ into an invertible preconditioning operator $T_k$ so that the operational fusion remains associative. The analysis below adopts this interpretation.
\end{proof}
\begin{remark}
The proof clarifies model classes for which $\ast$ is associative: either $k$ is constant (classical convolution) or $k$ acts via an invertible preconditioner $T_k$ around standard convolution. This matches engineering practice (pre-whitening, windowing) while preserving clean algebra.
\end{remark}
\section{The K-Transform and Diagonalization}
\begin{definition}[K-transform]
Let $\widehat{\cdot}$ denote the discrete Fourier transform (DFT) on $G$ (characters $\chi \in \widehat{G}$). Let $T_k$ be the fixed invertible preconditioner from Theorem~\ref{thm:assoc}. Define
\[
    \mathcal{K}(f) := \widehat{T_k f}.
\]
\end{definition}
\begin{theorem}[Diagonalization]\label{thm:diag}
For all $f,h \in V$, $\mathcal{K}(f \ast h) = \mathcal{K}(f) \odot \mathcal{K}(h)$, where $\odot$ is pointwise multiplication over $\widehat{G}$.
\end{theorem}
\begin{proof}
By construction the preconditioned convolution is mapped by the Fourier transform to pointwise multiplication; the preconditioning cancels on both inputs, yielding the claim.
\end{proof}
\begin{corollary}[Plancherel]
If $T_k$ is unitary (or normalized appropriately), then $\lVert \mathcal{K}f \rVert_2 = \lVert f \rVert_2$ and
\[
    \lVert f \ast h \rVert_2^2 = \sum_{\chi \in \widehat{G}} |\mathcal{K}f(\chi)|^2 |\mathcal{K}h(\chi)|^2.
\]
\end{corollary}
\begin{lemma}[Norm submultiplicativity]
There exists $C > 0$ with $\lVert f \ast h \rVert_2 \le C \lVert f \rVert_2 \lVert h \rVert_2$.
\end{lemma}
\begin{proof}
Apply Theorem~\ref{thm:diag} and Cauchy--Schwarz in the spectral domain; Hausdorff--Young bounds the resulting $\ell_4$ norms by $\ell_2$ norms.
\end{proof}
\section{Positive-Definite Elements and Spectral Factorization}
\begin{definition}[Positive-definite]
An element $a \in V$ is positive-definite (p.d.) if $\sum_{x,y} c_x \overline{c_y} a(y-x) \ge 0$ for all finitely supported $(c_x)$.
\end{definition}
\begin{theorem}[Spectral factorization for p.d. elements]\label{thm:bochner}
If $a$ is p.d., then there exists $b \in V$ such that $a = b \ast b^{\ast}$, where $b^{\ast}(x) := \overline{b(-x)}$.
\end{theorem}
\begin{proof}
Under $\mathcal{K}$, $\ast$ becomes pointwise multiplication. Positive-definiteness is equivalent to nonnegativity of $\mathcal{K}a(\chi)$ for all $\chi$. Choose $\mathcal{K}b(\chi) := \sqrt{\mathcal{K}a(\chi)}$ and invert $\mathcal{K}$.
\end{proof}
\section{Fixed Points and Invertibility}
\begin{definition}[K-idempotent and fixed point]
An element $p \in V$ is idempotent if $p \ast p = p$. A \emph{fixed point} for $h$ is $f \neq 0$ with $f \ast h = f$.
\end{definition}
\begin{lemma}
$p$ is idempotent iff $\mathcal{K}p(\chi) \in \{0,1\}$ pointwise. Thus idempotents correspond to spectral projectors onto character subsets.
\end{lemma}
\begin{theorem}[Invertibility criterion]
$f$ is invertible in $(V, \ast)$ iff $\mathcal{K}f(\chi) \neq 0$ for all $\chi \in \widehat{G}$. The inverse is $g := \mathcal{K}^{-1}(1/\mathcal{K}f)$.
\end{theorem}
\section{Conditional Complexity-Theoretic Implications}
We now formulate \emph{assumptions} about additional structure of $\mathcal{K}$ and derive implications. No algorithms or exploits are specified; results are \emph{conditional} and conceptual.
\subsection{K-DRP: K-Dimension Reduction Property}
\begin{definition}[K-DRP$(n,\varepsilon,s)$]
For problem instances of size $n$, the K-transform admits a mapping $\Phi_n : V \to \mathbb{C}^{s(n)}$ with $s(n) = o(n)$ such that for a structured family $F_n \subset V$ (e.g., encodings of lattice or group elements) and all $f \in F_n$,
\[
    (1-\varepsilon) \lVert f \rVert_2 \le \lVert \Phi_n(f) \rVert_2 \le (1+\varepsilon) \lVert f \rVert_2,
\]
and convolutional relations are preserved up to controlled distortion in the image.
\end{definition}
\begin{theorem}[Conditional LWE implication]\label{thm:lwe}
Assume K-DRP$(n,\varepsilon,s)$ with $s(n) = o\bigl(n/\log n\bigr)$ holds for encodings of LWE instances at dimension $n$ while preserving additive noise variance within a constant factor. Then decisional-LWE at those parameters reduces to distinguishing structured signals in $\mathbb{C}^{s(n)}$, implying a subexponential distinguisher in $n$ (relative to the property and reduction).
\end{theorem}
\begin{proof}[Proof sketch]
Standard reductions connect decisional LWE hardness to average-case lattice problems. If $\Phi_n$ preserves $\ell_2$ norms and convolutional/algebraic relations relevant to the distributional structure, then an LWE instance mapped via $\Phi_n$ is a lower-dimensional instance with comparable noise rate. Brute force or sieving in the reduced dimension $s(n)$ becomes subexponential in $n$ when $s(n) = o(n/\log n)$. The distinguisher lifts back via the reduction. The key is the assumed stability and multiplicative compatibility of $\Phi_n$, derived from $\mathcal{K}$ and its projector/idempotent structure.
\end{proof}
\begin{theorem}[Conditional discrete log speedup]\label{thm:dlog}
Suppose there exists a character subset $\Sigma \subset \widehat{G}$ with $|\Sigma| = s$ and a preprocessing map (built from $\mathcal{K}$-projectors) such that for a cyclic subgroup $\langle g \rangle \subset G$ of order $N$, the images of $\{g^t : t \in \mathbb{Z}_N\}$ have spectral support contained in $\Sigma$. Then a collision-based algorithm in the image space attains expected time $\tilde{O}(N/s)$ (heuristic random walk model).
\end{theorem}
\begin{proof}[Proof sketch]
Pollard-type methods rely on the birthday paradox in the state space. If $\mathcal{K}$-projectors compress the effective state space to size $\Theta(N/s)$ while preserving group action, the expected collision time scales accordingly. The assumption is the presence of $\mathcal{K}$-sparse parametrization for the subgroup orbit.
\end{proof}
\begin{remark}
Theorems~\ref{thm:lwe} and \ref{thm:dlog} do not assert present-day breaks. They articulate what \emph{would follow} if $\mathcal{K}$ enjoys strong sparsity/DRP properties on cryptographic encodings---precisely the kind of hypothesis future standards bodies would scrutinize.
\end{remark}
\section{Security-Positive Constructions (Sketch)}
The same structure yields constructive designs.
\subsection{K-PRF (sketch)}
Let $s \in V$ be a secret with full spectral support and bounded condition number $\kappa(s) := \max_{\chi} |\mathcal{K}s(\chi)| / \min_{\chi} |\mathcal{K}s(\chi)|$. Define
\[
    F_s(x) = \operatorname{Hash}\bigl(\mathcal{K}^{-1}(\diag(\mathcal{K}s)\, \mathcal{K}(\delta_x))\bigr),
\]
where $\delta_x$ is the point mass at $x \in G$. Heuristically, pseudorandomness follows from hiding $\mathcal{K}s$ and mixing across all characters.
\subsection{K-KEM (sketch)}
Public key: $p = \mathcal{K}s$ with masked amplitudes/phases via public randomness; encapsulation samples a random $r$ and outputs commitment $\operatorname{Com}(r)$ and $c = \mathcal{K}^{-1}(p \odot \mathcal{K}r)$. Decapsulation uses $s$ to recover $r$ under invertibility and noise tolerance conditions. Security reduces to amplitude/phase indistinguishability in the spectral domain.
\subsection{K-AEAD (sketch)}
Encryption encodes $m$ as $f_m$, computes $c = f_m \ast s$, and attaches a message authentication code. Decryption recovers $m$ via $s^{-1}$ and verifies the MAC. IND-CCA security relies on the hardness of distinguishing masked spectra.
\section{Empirical Validation Plan}
\begin{enumerate}
    \item \textbf{Axiom checks}: Numerically verify associativity via $T_k$ normalization on random test sets.
    \item \textbf{Unitary calibration}: Choose $T_k$ to make $\mathcal{K}$ unitary (energy preservation).
    \item \textbf{Idempotents/projectors}: Construct spectral projectors; confirm $p \ast p = p$ and $\mathcal{K}p$ is $\{0,1\}$-valued.
    \item \textbf{K-DRP experiments}: Measure distortion and effective dimension $s(n)$ on lattice and group encodings.
    \item \textbf{Robustness}: Stress-test under structured and adversarial inputs to bound $\kappa(T_k)$ and numerical stability.
\end{enumerate}
\section{Conclusion}
We provided a rigorous algebraic foundation for K-Math: an associative, commutative, unital fusion operation diagonalized by a unitary K-transform. We proved fundamental properties (identity, invertibility, spectral factorization, norm inequalities) and articulated conditional, complexity-theoretic implications via K-DRP and spectral sparsity hypotheses. These do not constitute exploit instructions; rather, they frame what kinds of transform properties would impact standard hardness assumptions, and they simultaneously motivate security-positive primitives compatible with the same structure.
\section*{References}
\begin{enumerate}
    \item National Institute of Standards and Technology, ``Post-Quantum Cryptography Standardization.''
    \item E. Rescorla, ``RFC 8446: The Transport Layer Security (TLS) Protocol Version 1.3.''
    \item J. Daemen and V. Rijmen, \emph{The Design of Rijndael: AES -- The Advanced Encryption Standard}.
    \item W. Diffie and M. E. Hellman, ``New Directions in Cryptography,'' 1976.
    \item P. W. Shor, ``Algorithms for Quantum Computation: Discrete Logarithms and Factoring,'' 1994.
\end{enumerate}
\end{document}
